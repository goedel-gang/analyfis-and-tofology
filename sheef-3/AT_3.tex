\documentclass[a4paper,12pt,fleqn]{article}
\author{Izaak \& Alastair}
\title{ANALYFIS AND TOFOLOGY - EXAMPLES 3}

\usepackage{mysty}
\usepackage{mymaths}

% Embed source files into PDF in case of loss. You can view or extract the
% source files by doing `pdfdetach -list <file.pdf>` or
% `pdfdetach -saveall <file.pdf>`, using pdfdetach from poppler, or some other
% suitable method.
\usepackage[main]{embedall}
\embedfile{mymaths.sty}
\embedfile{mysty.sty}

\begin{document}
\maketitle


\begin{enumerate}[label=\arabic*.,leftmargin=*]
 \item
  Firstly, note that since path-connectedness implies connectedness, it suffices
  in each case to either prove path connectedness, or disconnectedness.

  We write ``\(X\)'' to denote each respective space while we are talking about
  it. We will also write \(x \pcon y\) to denote the equivalence relation
  ``there exists a path from \(x\) to \(y\)''.
  \begin{enumerate}[label=(\roman*)]
   \item
    \(X\) is disconnected. Let \(U = D_1(-1, 0)\) and \(V = D_1(1, 0)\). Then
    \(U\) and \(V\) are clearly both open, disjoint, non-empty, and
    \(X = U \union V\) by definition.
   \item
    \(X\) is path-connected. Indeed, in \(\R^n\), if \(\vec y\) is any point in
    a ball\footnote{%
     Open or closed, or in fact any set lying in between the open and closed
     balls of radius \(r\) around \(\vec x\).
    } centred at \(\vec x\) - call the ball \(B\) of radius \(r\) - then there
    is a path from \(\vec y\) to \(\vec x\), given by the line
    \(\vec \gamma: t \mapsto \vec y + t(\vec x - \vec y)\).
    \begin{itemize}
     \item
      \(\vec \gamma\) is continuous eg because
      each component \(q_i \compose \vec \gamma: t \mapsto y_i + t(x_i - y_i)\)
      is continuous.

      Alternatively, we can directly show that \(\vec \gamma\) is
      \(\norm{\vec x - \vec y}\)-Lipschitz in the Euclidean metric.
     \item
      We also need to show that \(\vec \gamma\) is well-defined.
      \(\vec \gamma(0) = \vec y \in B\) by choice of \(\vec y\). If
      \(0 < t \le 1\), then
      \(d(\vec \gamma(t), \vec x)
        = (1 - t)\norm{\vec x - \vec y}
        < \norm{\vec x - \vec y}
        \le r\).
      It follows that \(\vec \gamma(t) \in B\).
    \end{itemize}
    Lastly observe that in \(X\), the points \((-1, 0)\) and \((1 ,0)\) are
    connected by the line \(\vec \gamma: t \mapsto (-1, 0) + t(2, 0)\). Again,
    this is clearly continuous, and for \(0 \le t < \tfrac 12\), we have
    \(\vec \gamma(t) \in D_1(-1, 0)\), and for \(\tfrac 12 \le t \le 1\), we
    have \(\vec \gamma(t) \in B_1(-1, 0)\).

    \(\pcon\) is an equivalence relation, so by the above, \(X\) is
    path-connected:

    Since there is a path from each \(\vec x \in X\) to \((-1, 0)\) or
    \((1, 0)\), so \(\vec x \pcon (1, 0)\), by transitivity and symmetry, all
    points are path-connected to each other.
   \item
    Let \((x, y) \in X\). We show that \((x, y) \pcon (0, 0)\).

    Define the path \(\vec \gamma: t \mapsto (1 - t)(x, y)\). Clearly
    \(\vec \gamma\) is continuous.
    \begin{itemize}
     \item
      If \(x = 0\), then \((1 - t)x = 0\) for all \(t \in \intcc{0, 1}\), so
      \(\vec \gamma\) is well-defined.
     \item
      If \(x \ne 0\) and \(y / x \in \Q\), then for \(0 \le t < 1\), we have
      \([(1 - t)y] / [(1 - t)x] = y / x \in \Q\), and when \(t = 1\),
      \((1 - t)x = 0\). So \(\vec \gamma\) is well-defined.
    \end{itemize}
    By the same equivalence relation argument as before, \(X\) is
    path-connected.
   \item
    \(X\) is disconnected. Let
    \begin{align*}
     U &= \set{(x, y) : x > \sqrt 2 y} \subset \R^2 \\
     V &= \set{(x, y) : x < \sqrt 2 y} \subset \R^2
    \end{align*}
    Defining the continuous function \(f(x, y) = \sqrt 2 y - x\), it is clear
    that \(U = f^{-1}(-\infty, 0)\) and \(V = f^{-1}(0, \infty)\) are open.

    Furthermore, \((U \union V)^C\) does not intersect with \(X\),
    since if \((x, y) \in (U \union V)^C\), then \(\sqrt 2 y = x\), so if
    \(x = 0\), then \(y = 0\), so \((x, y) \notin X\). But if
    \(x \ne 0\), then \(y / x = \tfrac 12 \sqrt 2 \notin \Q\), so
    \((x, y) \notin X\).

    So \(X \subset U \union V\). But clearly \(U \intersect V = \emptyset\), so
    \(U \intersect X\) and \(V \intersect X\) disconnect \(X\).
  \end{enumerate}
 \item
  Let \(T \subset S\). We show that \(f^{-1}(T)\) is open in \(X\).

  Indeed, let \(x \in f^{-1}(T)\). Let \(U\) be a neighbourhood of \(X\) on
  which \(f\) is constant. Then for any \(u \in U\),
  \(f(u) = f(x) \in T\), so \(u \in f^{-1}(T)\), ie \(U \subset f^{-1}(T)\).
  This proves \(f^{-1}(T)\) is open.

  Now suppose \(f\) is not constant, ie there are distinct \(x_1\) and \(x_1\)
  such that \(f(x_1) \ne f(x_2)\). Let \(T_1 = \set{x_1}\), and
  \(T_2 = S \setminus \set{x_1}\). Let \(U = f^{-1}(T_1)\) and
  \(V = f^{-1}(T_2)\).

  Then \(x_1 \in U\) and \(x_2 \in V\), so both are non-empty. By the above,
  they are open. We have
  \(U \union V = f^{-1}(T_1 \union T_2) = f^{-1}(S) = X\), and
  \(U \intersect V = f^{-1}(T_1 \intersect T_2) = f^{-1}(\emptyset) =
  \emptyset\).

  So \(U\) and \(V\) disconnect \(X\), which is a contradiction.
 \item
  Let \(X\) and \(Y\) be tofological spaces, and let \(f: X \to Y\) be a
  homeomorphism. Write \(\mathcal C(X)\) and \(\mathcal C(Y)\) for the
  respective sets of connected components. We claim that
  \begin{alignat*}2
   f&:& \mathcal C(X) &{}\to \mathcal C(Y) \\
   &&S &{}\mapsto f(S) \\
   &&f^{-1}(S') &{}\mapsfrom S'
  \end{alignat*}
  is a bijection.

  Indeed, let \(S \subset X\) be connected. Since \(f\) is continuous and \(S\)
  is connected, \(f(S)\) is connected. Now suppose \(f(S)\) is not a maximal
  connected set - ie, there is some connected \(Y' \subset Y\) with
  \(f(S) \subsetneq Y'\). Since \(f^{-1}\) is continuous, \(f^{-1}(Y')\) is
  connected, and since \(f\) is a bijection, \(S \subsetneq f^{-1}(Y')\). So
  \(S\) is not a maximal connected set, which is a contradiction. So \(f(S)\)
  was a connected component.

  That \(f^{-1}\) also sends connected components to connected
  components follows from the fact that \(f^{-1}\) is also a homeomorphism.

  It follows that \(\card{\mathcal C(X)} = \card{\mathcal C(Y)}\).

  A cheeky optimisation here is to use the fact that \(\intcc{0, 1}\) is
  compact, but the other aren't compact. We give a more principled argument.

  For a tofological space, define \(N_1(X)\) to be the set
  \(\set{x \in X : \text{\(X \setminus \set{x}\) is connected}}\).

  The size of this set is well-defined up to homeomorphism: if
  \(f: X \to Y\) is a homeomorphism, then
  \(f\restrict_{X \setminus \set x}:
    X \setminus \set x \to Y \setminus \set{f(x)}\) is a homeomorphism, so
  particularly, if \(x \in N_1(X)\), then \(f(x) \in N_1(Y)\). It follows
  that \(x \in N_1(X) \iff f(x) \in N_1(Y)\), ie \(f\) restricts to a
  bijection \(f\restrict_{N_1(X)}: N_1(X) \to N_1(Y)\).

  But clearly \(N_1\intoo{0, 1} = \emptyset\) and
  \(N_1\intco{0, 1} = \set 0\) and \(N_1\intcc{0, 1} = \set{0, 1}\) all have a
  different number of elements. So these three are not homeomorphic.

  There is another invariant we could have used. Namely, it is possible to
  remove two points from \(\intcc{0, 1}\) without disconnecting it, but if we
  remove two points \(x\) and \(y\) from \(X = \intco{0, 1}\) or
  \(X = \intoo{0, 1}\), then we must have either \(x\) or \(y\) being nonzero -
  say \(0 < x < 1\), without loss of generality.
  Then \(U = (-\infty, x) \intersect (X \setminus {x, y})\) and
  \(V = (x, \infty) \intersect (X \setminus{x, y})\)
  disconnect \(X \setminus \set{x, y}\).

  Similarly, we can remove a point from \(\intco{0, 1}\) without disconnecting
  it, but this cannot be done for \(\intoo{0, 1}\).

  For \(\mathsf A\) and \(\mathsf H\), a nice invariant to use is the fact that
  it is possible to remove a single point from \(\mathsf H\) that disconnects it
  into three components, but there is no such point in \(\mathsf A\).

  Alternatively, \(\abs{N_1(\mathsf H)} = 4\) - it just consists of the
  end-points of each line. But \(N_1(\mathsf A)\) has at least continuum many
  elements, namely at least each point lying inside the loop in the middle of
  \(\mathsf A\).

  A slightly messier alternative is that fact that is is possible to remove
  \(4\) points from \(\mathsf H\) without disconnecting it, but this cannot be
  done for \(\mathsf A\). This one is fairly obvious if you think about it but
  it doesn't feel very satisfying to me.

  \begin{question}
   Can we characterise the end-points of each line tofologically, in an
   elementary-ish way? This would lead to another nice proof if that's possible,
   since \(\mathsf A\) only has \(2\) such points, but \(\mathsf H\) has \(4\).
  \end{question}
 \item
  Firstly, we claim that \(C_n = \set{1/n} \times \intcc{0, 1}\) is a connected
  component. Clearly it is connected. Moreover, let
  \(U_n
     = \bracks[\big]{\intoo[\big]{
        \tfrac 12(\tfrac 1n + \tfrac 1{n + 1}),
        \tfrac 12(\tfrac 1n + \tfrac 1{n - 1})}
     \times \intoo{-1, 2}} \intersect X\),
  where we take \(1 / 0 = \infty\). Let
  \(V_n
     = X \setminus \bracks[\big]{\intcc{
        \tfrac 12(\tfrac 1n + \tfrac 1{n + 1}),
        \tfrac 12(\tfrac 1n + \tfrac 1{n - 1})}
     \times \intcc{-1, 2}} \intersect X\).
  Then clearly \(X = U_n \union V_n\) is a union of disjoint non-empty open
  sets. In fact, \(U_n = C_n\). Suppose that \(C_n\) is not a maximal
  connected set, ie there is some connected \(C'\) with \(C_n \subsetneq C'\).
  Then \(U_n\) and \(C' \intersect V_n\) disconnect \(C'\), which is a
  contradiction. So \(C_n\) is a connected component.

  Lastly, \(\set{(0, 0)}\) and \(\set{(0, 1)}\) must be separate connected
  components, since \(\set{(0, 0), (0, 1)}\) is disconnected.

  Now let \(x = (0, 0)\) and \(y = (0, 1)\). Suppose \(U\) and \(V\) disconnect
  \(X\), and \(x \in U\), and \(y \in V\). Note that if \(U\) intersects with
  some connected component \(C\), then \(U\) must contain all of \(C\)
  (otherwise \(U \intersect C\) and \(V \intersect C\) would disconnect \(C\)).

  By openness of \(U\) and \(V\), choose \(\delta_1, \delta_2 > 0\) such that
  \(D_{\delta_1}^X(x) \subset U\) and \(D_{\delta_2}^X(y) \subset V\).

  Choose \(N > [\min \set{\delta_1, \delta_2}]^{-1}\). Then \((1/N, 0) \in U\),
  so \(U\) intersects with \(C_N\), so \(C_N \subset U\).
  But also \((1/N, 1) \in V\), so \(C_N \subset V\). So
  \(C_N \subset U \intersect V\), contradicting disjointness.
 \item
  We argue by Heine-Borel.

  Firstly note that \(\norm{\cdot}: A \to \R\) is continuous, so it must be
  bounded. So \(A\) is bounded.

  Now suppose \(A\) is not closed. Let \(a \in \cl(A) \setminus A\). Then define
  \begin{align*}
   f: A &\to \R \\
   x &\mapsto 1 / d(x, a)
  \end{align*}
  Since \(x \mapsto d(x, a)\) is continuous and positive
  (eg by sheet 2, question 5: \(d(x, a) = d(x, \set a)\)), and
  \(x \mapsto 1 / x\) is a continuous function on the positive reals, \(f\) is
  continuous. So \(f\) must be bounded.

  But also by sheet 2 question 5, there is some sequence \((x_n)\) in \(A\) with
  \(x_n \to a\), so \(d(x_n, a) \to 0\), so \(f(x_n) \to \infty\), contradicting
  boundedness. So \(A\) is closed, and therefore compact.
 \item
  WLOG we will take \(A\) to be bounded. Particularly \(A\) is compact, so
  sequentially compact.

  Suppose \((a_n + b_n)\) is a sequence in \(A + B\), with \(a_n + b_n \to x\)
  in \(\R^n\).

  Let \((a_{k_n})\) be a convergent subsequence of \((a_n)\). Say
  \(a_{k_n} \to a\) in \(A\).

  Then \(b_{k_n} = a_{k_n} + b_{k_n} - a_{k_n} \to x - a\) in \(\R^n\). But
  \(B\) is closed, so \(x - a \in B\).

  So \(x = a + (x - a) \in A + B\), and \(A + B\) must be closed.

  For our counterexample, let \(A = \N\), and
  \(B = \set{-n + 1 / (n + 1) : n \in \N}\) (here we take \(0 \notin \N\)).
  Then \(1 / (n + 1) = n + -n + 1 / (n + 1) \in A + B\) for any \(n\), but since
  \(n + 1 \ge 2\), we have \(0 < 1 / (n + 1) \le \tfrac 12\), so
  \(B \intersect \Z = \emptyset\) and particularly \(0 \notin A + B\).

  To show that \(A\) and \(B\) are closed, we can simply write down their
  complements:
  \begin{align*}
   \R \setminus A
    &= (-\infty, 1) \union (1, 2) \union (2, 3) \union \dotsb \\
   \R \setminus B
    &= (-\tfrac 12, \infty) \union (-\tfrac 53, -\tfrac 12)
       \union (-\tfrac{11}{12}, -\tfrac 53) \union \dotsb
  \end{align*}
  both of which are clearly open. This can be extended to any diverging monotone
  sequence in \(\R\), or more generally still, any sequence having no convergent
  subsequence.

  Alternatively, we could argue that if \(n \ne m\), then
  \begin{align*}
   \abs[\big]{-n + \tfrac 1{n + 1} + m - \tfrac 1{m + 1}}
    &\ge \abs{m - n} - \abs[\big]{\tfrac 1{n + 1} - \tfrac 1{m + 1}} \\
    &\ge \abs{m - n} - \tfrac 1{n + 1} - \tfrac 1{m + 1} \\
    &\ge 1 - \tfrac 12 - \tfrac 13 \\
    &= \tfrac 16
  \end{align*}
  and \(\abs{n - m} \ge 1\), so any convergent sequence in \(A\) (or resp.
  \(B\)) must be eventually constant.
 \item
  Define the spherical polar parametrisation
  \begin{alignat*}2
   f&:& Q &\to S^2 \\
   && (s, t) &\mapsto
   \begin{pmatrix}
    \cos 2 \pi s \sin \pi t \\
    \sin 2 \pi s \sin \pi t \\
    \cos \pi t
   \end{pmatrix},
  \end{alignat*}
  and establish that \(f\) is well-defined and fully respects \(R\), \(f\) is
  surjective, and \(f\) is continuous. Extract a continuous bijection
  \(\tilde f: Q/R \to S^2\), which must be a homeomorphism by the tofological
  inverse function theorem. The algebra is provided in the appendix.
  % or is it dummy
 \item
  Suppose \(X\) is a sequentially compact tofological space, and \(f: X \to \R\)
  is continuous. Suppose \(f\) is unbounded. Then we can choose an \(x_n\) with
  \(\abs{x_n} \ge n\), for all \(n \in \N\). Then no subsequence of \(f(x_n)\)
  is bounded, so no subsequence converges. This contradicts sequential
  compactness of \(X\), since if \(x_{k_n}\) is a convergent subsequence, then
  \(f(x_{k_n})\) must also converge.

  Now we prove \(f\) attains its upper bound. Let \(M = \sup_X f\).
  For each \(n \in \N\), pick an \(x_n\) such that
  \(f(x_n) \in \intcc{M - \tfrac 1n, M}\) (this is possible as otherwise \(M\)
  would not be the supremum).

  Take a convergent subsequence \(x_{k_n} \to x\) in \(X\).
  Then \(f(x_{k_n}) \to f(x)\). But by construction, we have
  \(\abs{f(x_{k_n}) - M} \le 1/k_n \to 0\), so \(f(x_{k_n}) \to M\).
  So \(f(x) = M\). The argument to show that \(f\) attains its infimum is
  similar.

  Now suppose \((M, d)\) is a compact metric space, \((M', d')\) is any metric
  space, and let \(f: M \to M'\) be continuous. Fix an \(\epsilon > 0\). For
  each \(x \in M\), choose \(\delta_x\) such that
  \(d(x, y) < \delta_x \implies d'(f(x), f(y)) < \tfrac 12 \epsilon\).

  Let \(\mathcal U = \set{D_{\delta_x / 2}(x) : x \in M}\). Clearly
  \(\mathcal U\) is an open cover of \(M\), so pick a finite subcover
  \(\mathcal V
    = \set{D_{\delta_{x_1} / 2}(x_1), \dotsc, D_{\delta_{x_n} / 2}(x_n)}\).
  Let \(\delta = \min_{1 \le i \le n} \tfrac 12 \delta_{x_i}\).

  Then suppose \(x, y \in M\) with \(d(x, y) < \delta\).
  Since \(\mathcal V\) covers \(M\), we can find an \(x_i\) such that
  \(d(x, x_i) < \tfrac 12 \delta_{x_i}\). Then
  \(d(y, x_i)
    \le d(y, x) + d(x, x_i)
    < \tfrac 12 \delta_{x_i} + \delta
    < \delta_{x_i}\).
  So both \(x\) and \(y\) lie in \(D_{\delta_{x_i}}(x_i)\), and we can apply the
  triangle inequality one last time:
  \(d(f(x), f(y))
    \le d(f(x), f(x_i)) + d(f(x_i), f(y))
    < \tfrac 12 \epsilon + \tfrac 12 \epsilon
    = \epsilon\).
  So \(f\) is uniformly continuous!
  \begin{remark}
   The really crucial step here was to make our cover \(\mathcal U\) not consist
   of full \(\delta_x\)-balls, but of \(\tfrac 12 \delta_x\)-balls. This meant
   that when we had our arbitrary close-together \(x\) and \(y\) later on, we
   wouldn't run into the problem of having \(x\) and \(y\) being very close
   together but still in different \(\delta_x\)-balls.

   Aside from that, it was a fairly standard two-epsilon compactness proof.
   It's a nice example of how compactness lets us turn local properties into
   global properties by reducing to a finite number of areas.
  \end{remark}
  \begin{question}
   Is there some sort of clever way to do this by recycling a previous
   compactness result? Maybe writing down a clever continuous function involving
   \(d\)?
  \end{question}
 \item
  \begin{enumerate}[label=(\alph*)]
   \item
    Let \(\equivr\) be an equivalence relation on a set \(S\), and \(f: T \to S\)
    be a function. Then
    \(R = \set{(t_1, t_2) \in T \times T : f(t_1) \equivr f(t_2)}\) is an
    equivalence relation on \(T\), since
    \begin{itemize}
     \item
      \(f(t) \equivr f(t)\) for all \(t\), by reflexivity of \(\equivr\).
     \item
      If \(f(t_1) \equivr f(t_2)\), then \(f(t_2) \equivr f(t_1)\) by symmetry of
      \(\equivr\).
     \item
      If \(f(t_1) \equivr f(t_2)\) and \(f(t_2) \equivr f(t_3)\),
      then \(f(t_1) \equivr f(t_3)\), by transitivity of \(\equivr\).
    \end{itemize}
    It follows that the given \(R\) is an equivalence relation on \(X\).

    Now, deviating from the question's notation, let \(q: X \to X/R\),
    \(q_1: X \to X / R_1\) and \(q_2: X / R_1 \to (X / R_1) / R_2\) be the
    quotient maps. Note that
    \begin{alignat*}2
     && (q_2 \compose q_1)(x) &{}= (q_2 \compose q_1)(y) \\
     &\iff{}& q_2(q_1(x)) &{}= q_2(q_1(y)) \\
     &\iff{}& (q_1(x), q_1(y)) &{}\in R_2 \\
     &\iff{}& (x, y) &{}\in R
    \end{alignat*}
    So \(q_2 \compose q_1\) fully respects \(R\). Clearly \(q_2 \compose q_1\) is
    continuous, and as a composition of surjective functions it is also
    surjective.

    So there is a continuous bijection \(f: X / R \to (X / R_1) / R_2\), such that
    \(f \compose q = q_2 \compose q_1\).

    Now note that the function \(q: X \to X/R\) respects the relation \(R_1\),
    inducing a continuous surjection \(g: X/R_1 \to X/R\), with
    \(g \compose q_1 = q\). Now \(g\) fully respects \(R_2\), since
    \begin{alignat*}2
     && g([x]_{R_1}) &= g([y]_{R_1}) \\
     &\iff{}& g(q_1(x)) &= g(q_1(y)) \\
     &\iff{}& q(x) &= q(y) \\
     &\iff{}& (x, y) &\in R \\
     &\iff{}& ([x], [y]) &\in R_2
    \end{alignat*}
    so we obtain a continuous bijection \(\tilde g: (X / R_1) / R_2 \to X / R\)
    such that \(\tilde g \compose q_2 = g\).

    Now observe that
    \(\tilde g(f([x]_R))
      = \tilde g(q_2(q_1(x)))
      = g(q_1(x))
      = q(x))
      = [x]\), and also\footnote{%
     Strictly, since we already know \(f\) is a bijection, we could have skipped
     the second calculation.
    } that
    \(f(\tilde g([[x]_{R_1}]_{R_2}))
      = f(\tilde g(q_2(q_1(x))))
      = f(g(q_1(x)))
      = f(q(x)))
      = q_2(q_1(x))
      = [[x]_{R_1}]_{R_2}\).
    This shows that \(\tilde g\) is in fact the inverse of \(f\), and since
    \(\tilde g\) is continuous, this shows \(f\) is a homeomorphism.
    \begin{question}
     Could we have just stopped after obtaining \(f\) and have showed that \(f\)
     is an open map directly? ie, how necessarily fiddly would that be?
    \end{question}
   \item
    The conceit of this part of the question is to write both the sphere and the
    torus as the parametrisations we have on the square \(Q\), and then
    establish that the ``composite'' equivalence relations after quotienting out
    the subsets \(A\) and \(B\) are the same (or give obviously homeomorphic
    spaces). The algebra is again given in the appendix.
    % or is it dummy
  \end{enumerate}
 \item
  \begin{enumerate}[label=(\alph*)]
   \item \label{q12a}
    Let \(W \subset X \times Y\) be open. It suffices to prove that
    \(\pi_X(W)\) is open in \(X\). So suppose \(x \in \pi_X(W)\). Then there is
    some \(y \in Y\) such that \((x, y) \in W\). Since \(W\) is open, we can
    pick open \(U, V\) such that \(x \in U \times V \subset W\). Then
    \(x \in U \subset \pi_X(W)\), so \(\pi_X(W)\) is indeed open.

    Now suppose \(Y\) is compact, \(F \subset X \times Y\) is closed, and let
    \(x \notin \pi_X(F)\). For each \(y \in Y\), choose open sets \(U_y\) and
    \(V_y\) such that \(x \subset U_y \times V_y \subset F^C\), by closure of
    \(F\).

    Then \(\mathcal V = \set{V_y : y \in Y}\) is an open cover of \(Y\). So by
    compactness, we can take a finite subcover
    \(\mathcal W = \set{V_{y_1}, \dotsc, V_{y_n}}\). Let
    \(U = \Intersect_{i = 1}^n U_{y_i}\). It is clear that \(x \in U\), and that
    \(U \subset X\) is open.

    Furthermore, if \(x' \in \pi_X(F)\), then there is some \(y' \in Y\) such
    that \((x', y') \in F\). Then \(y' \in V_{y_i}\) for some \(i\), as
    \(\mathcal W\) covers \(Y\). Since \((x', y') \in F\) and
    \(U_{y_i} \times V_{y_i} \subset F^C\), we have \(x' \notin U_{y_i}\), so
    \(x' \notin U\). This shows \(U \subset F^C\) and that therefore \(F\) is
    closed.

    Let \(f: \R^2 \to \R\) be given by \(f(x, y) = xy\), and let
    \(F = f^{-1}\set 1\) (the graph of \(y = 1/x\), if you will). Since \(f\) is
    continuous, \(F\) is closed. However,
    \(\pi_X(F) = \pi_Y(F) = \R \setminus \set 0\) is not closed.
   \item
    Suppose \(f: X \to Y\) is continuous, \(Y\) is Hausdorff, and \(\Gamma\) is
    closed in \(X \times Y\). Suppose \((x, y) \notin \Gamma\). Then
    \(y \ne f(x)\), so by Hausdorffness, we can choose disjoint open
    \(U, V \subset Y\) such that \(f(x) \in U\) and \(y \in V\). Then
    \(f^{-1}(U)\) is open and \((x, y) \in f^{-1}(U) \times V\). Note that if
    \((x', y') \in \Gamma\) (ie, \(y' = f(x')\)), then we cannot have
    \((x', y') \in f^{-1}(U) \times V\). If we did, then
    we would have \(x' \in f^{-1}(U)\) and \(y' \in V\). But then
    \(f(x') \in U\), so \(y' \in U \intersect V\), contradicting disjointness.

    So \((x, y) \in f^{-1}(U) \times V \subseteq \Gamma^C\), which proves
    \(\Gamma\) is closed.

    Now conversely suppose \(\Gamma\) is closed and \(Y\) is compact. Let
    \(V \subset Y\) be closed. Then \(\pi_Y^{-1}(V) \intersect \Gamma\) is
    closed, so by \ref{q12a}, we have that
    \(\pi_X(\pi_Y^{-1}(V) \intersect \Gamma)\) is closed. Now we claim that this
    set is precisely \(f^{-1}(V)\). It follows that \(f\) is continuous.

    For the last fact, firstly suppose that \(x \in f^{-1}(V)\).
    Then \(f(x) \in V\), so \((x, f(x)) \in \Gamma\), and
    \((x, f(x)) \in \pi_Y^{-1}(V)\), so
    \((x, f(x)) \in \pi_Y^{-1}(V) \intersect \Gamma\). So
    \(x \in \pi_X(\pi_Y^{-1}(V) \intersect \Gamma)\).

    Conversely, if \(x \in \pi_X(\pi_Y^{-1}(V) \intersect \Gamma)\), there
    is a \(y \in Y\) such that
    \((x, y) \in \pi_Y^{-1}(V) \intersect \Gamma\). Since \((x, y) \in \Gamma\),
    we have \(y = f(x)\). But also, since \((x, y) \in \pi_Y^{-1}(V)\), we have
    \(y \in V\). So \(f(x) \in V\), so \(x \in f^{-1}(V)\).
    \begin{remark}
     The author recommends drawing a picture of what's going on with the
     constructions in both parts, with a not-too-pathological, preferably
     monotone, continuous function on \(\R\).
    \end{remark}
  \end{enumerate}
 \item
  \begin{enumerate}[label=(\alph*)]
   \item
    \textit{Existence}. Define \(g: M \to \R\) by \(g(x) = d(x, f(x))\). \(g\)
    is continuous, since \(d: M \times M \to \R\) is continuous (for example,
    check that \(d: M \oplus_1 M \to \R\) is \(1\)-Lipschitz).

    Since \(M \ne \emptyset\), we know that \(g\) must have a lower bound, and
    attain it. Say \(k = \inf_M g\) and \(g(x) = d(x, f(x)) = k\). Certainly
    \(k \ge 0\), as \(g\) is positive. If \(k > 0\), then \(x \ne f(x)\), so
    \(d(f(x), f(f(x))) < d(x, f(x)) = k\), so in fact \(g(f(x)) < k\), so \(k\)
    was not minimal.

    Therefore \(k = 0\), so \(x = f(x)\) and we have a fixed point.

    \textit{Uniqueness}. Suppose \(x \ne y\) are fixed points. Then
    \(d(x, y) = d(f(x), f(y)) < d(x, y)\), which is a contradiction.
   \item
    Fix \(y \in M\). Define \(g: M \to \R\) by \(g(x) = d(f(x), y)\). Clearly
    \(g\) is continuous, and it attains its infimum once more. Say
    \(k = \inf_M g\) and \(g(x) = k\). If \(k = 0\), then \(f(x) = y\), which
    proves that \(f\) is surjective. So suppose \(k > 0\).

    Write \(f^n\) for the \(n\)-fold composite of \(f\) (so \(f^1 = f\) and
    \(f^{n + 1} = f \compose f^n\) (also take \(f^0 = \Id\))). Now since \(k\)
    is the infimum of \(g\), we know \(g(f^n(y)) \ge k\) for all \(n \ge 0\), so
    \(d(f^n(y), y) \ge k\) for all \(n \in \N\). Applying isometry \(m\)
    times yields \(d(f^{n + m}(y), f^m(y)) \ge k\) for any \(m, n \in \N\) - ie,
    for any \(m', n' \in \N\), if \(m' \ne n'\), then
    \(d(f^{m'}(y), f^{n'}(y)) \ge k\).

    Particularly, this shows that the sequence \((f^n(y))\) in \(M\) has no
    Cauchy subsequence, so certainly no convergent subsequence. So \(M\) is not
    sequentially compact, which is a contradiction.
  \end{enumerate}
 \item
  \begin{enumerate}[label=(\alph*)]
   \item
    Let \(X\) be compact Hausdorff, and \(A, B \subset X\) be closed. Note that
    \(A\) and \(B\) must both be compact, since they are closed in \(X\).

    For all
    \(a \in A\) and \(b \in B\), choose open sets \(U_{ab}, V_{ab} \subset X\)
    separating \(a\) and \(b\), by Hausdorffness
    (ie, such that \(a \in U_{ab}\) and \(b \in V_{ab}\) and
    \(U \intersect V = \emptyset\)).

    By construction, for any fixed \(a \in A\),
    \(\mathcal V_a = \set{V_{ab} : b \in B}\) is an open cover of \(B\).
    So take a finite subcover
    \(\mathcal V_a' = \set{V_{ab_1}, \dotsc, V_{ab_n}}\).
    Let \(U_a = \Intersect_{i = 1}^n U_{ab_i}\) and let
    \(V_a = \Union_{i = 1}^n V_{ab_i}\). Then both \(U_a\) and
    \(V_a\) are open, and we have \(a \in U_a\), and
    \(B \subset V_a\). Note also that \(U_a \intersect V_a = \emptyset\), since
    \(V_{ab_i} \intersect U_a = \emptyset\) for all \(i\).

    Now, letting \(a\) vary, observe that \(\mathcal U = \set{U_a : a \in A}\)
    is an open cover of \(A\). Take a finite subcover
    \(\mathcal U' = \set{U_{a_1}, \dotsc, U_{a_k}}\).
    Let \(U = \Union_{i = 1}^k U_{a_i}\) and let
    \(V = \Intersect_{i = 1}^k V_{a_i}\).
    Now we have that \(U\) and \(V\) are both open, and \(A \subset U\),
    and \(B \subset V\), and since \(U_{a_i} \intersect V = \emptyset\) for all
    \(a\), we have \(U \intersect V = \emptyset\).

    So \(X\) is normal.
    \begin{remark}
     Really what happened here is we built up the amount of separation we could
     get. The first step was to separate a point from any arbitrary closed set,
     after which we could basically think of \(B\) as a point, and just do the
     same thing for \(A\).

     The real hypothesis we needed was that \(A\) and \(B\) are compact. In some
     sense this shows that compact sets behave a bit like points inside a
     Hausdorff space, another example of which is the fact that compact subsets
     of Hausdorff spaces are closed.
    \end{remark}
   \item
    We will be thinking of the ambient space as \(C_1\), which is Hausdorff and
    compact. Define \(C = \Intersect_{n = 1}^\infty C_n\), and note that \(C\)
    is closed, as an intersection of closed sets (each \(C_n\) is closed, as a
    compact set in a Hausdorff space).

    So suppose \(C\) is disconnected by sets \(A\) and \(B\):
    \(C = A \union B\). Then both \(A\) and \(B\) are closed in \(C\), which is
    a closed subset of \(C_1\). Therefore, \(A\) and \(B\) are both closed in
    \(C_1\). So we can choose disjoint open sets \(U, V \subset C\) such that
    \(A \subset U\) and \(B \subset V\).

    Define \(X_n = C_n \setminus (U \union V)\). Then \(X_n\) is closed
    in \(C_1\), and \((X_n)\) is a decreasing sequence of sets. Observe that
    \(\Intersect_{n = 1}^\infty X_n
      = \Intersect_{n = 1}^\infty C_n \setminus (U \union V)
      = C \setminus (U \union V) = \emptyset\), so taking complements,
    \(\Union_{n = 1}^\infty C_1 \setminus X_n = C_1\), ie we have
    \(\mathcal U = \set{C_1 \setminus X_n : n \in \N}\) being an open cover of
    \(C_1\). Take a finite subcover
    \(\mathcal V = \set{C_1 \setminus X_{m_1}, \dotsc, C_1 \setminus X_{m_k}}\),
    so that we have \(\Union_{n = 1}^k C_1 \setminus X_{m_n} = C_1\). Taking
    complements again, \(\Intersect_{n = 1}^k X_{m_n} = \emptyset\). But since
    \((X_n)\) is decreasing, if we set \(M = \max_{1 \le n \le k} m_n\), we know
    \(\Intersect_{n = 1}^k X_{m_n} = X_M = \emptyset\). So
    \(C_M \setminus (U \union V) = \emptyset\), ie
    \(C_M \subset (U \union V)\). But then
    \(U \intersect C_M\) and \(V \intersect C_M\) disconnect \(C_M\), since
    neither of these can be empty, because \(A \subset U \intersect C_M\) and
    \(B \subset V \intersect C_M\).
    This contradicts connectedness of \(C_M\).
    \begin{remark}
     The slightly weird step of taking complements here was really motivated by
     an equivalent condition to compactness, which is the \emph{finite
     intersection property}. This says that whenever there is a collection of
     closed sets in \(X\) whose intersection is empty, there must be some finite
     subset of those closed sets whose intersection is still empty. The author
     met this fact on one of Gareth's Met \& Top sheets. The proof is
     essentially what happens in the above proof before we appeal to
     compactness.
    \end{remark}
    For our counterexample, let
    \(C_n = \R^2 \setminus [\intoo{-1, 1}  \times \intoo{-\infty, n}]\). Then we
    have the the intersection is
    \(\Intersect_{n = 1}^\infty C_n
      = \R^2 \setminus [\intoo{-1, 1} \times \R]
      = [\intoc{-\infty, -1} \times \R] \times [\intco{1, \infty} \times \R]\),
    which is clearly disconnected.
  \end{enumerate}
 \item
  For separability, first define the set \(A_n \subset C\intcc{0, 1}\) to be
  \emph{ the set of continuous functions \(f\) on \(\intcc{0, 1}\) such that
  \(f\) is piecewise linear on \(\intcc{\tfrac in, \tfrac {i + 1}n}\) for each
   \(0 \le i < n\), \textbf{and} \(f(\tfrac in) \in \Q\) for each
   \(0 \le i \le n\)
  }.
  In symbols, \(f \in A_n\) if and only if
  \begin{align}
   f(\tfrac in) &\in \Q && \Forall 0 \le i \le n \label{A_n_Q} \\
   f(\tfrac{i + t}n) &= (1 - t)f(\tfrac in) + tf(\tfrac{i + 1}n)
    && \Forall 0 \le i < n, \Forall t \in \intcc{0, 1} \label{A_n_linear}
  \end{align}
  Clearly each \(A_n\) is countable, since we can inject it into the countable
  set \(\Q^{n + 1}\) via
  \begin{align*}
   \theta_n: A_n &\to \Q^{n + 1} \\
   f &\mapsto (f(0), f(\tfrac 1n), \dotsc, f(1 - \tfrac 1n), f(1))
  \end{align*}
  This is well-defined by (\ref{A_n_Q}), and an injection\footnote{%
   Functional intensionalists turn back now!
  }, since (\ref{A_n_linear}) allows us to recover \(f(x)\) for each
  \(x \in \intcc{0, 1}\) from \(\theta_n(f)\).

  It follows that \(A = \Union_{n = 1}^\infty A_n\) is also countable, as a
  countable union of countable sets. It only remains to show that \(A\) is dense
  in \(C\intcc{0, 1}\). To prove this, it suffices to show that every non-empty
  open \(C\intcc{0, 1}\)-ball intersects \(A\). So let \(f \in C\intcc{0, 1}\)
  and \(r > 0\) be arbitrary.

  Since \(\intcc{0, 1}\) is compact, \(f\) is uniformly continuous, so choose a
  \(\delta > 0\) such that whenever \(\abs{x - y} < \delta\), we have
  \(\abs{f(x) - f(y)} < \tfrac 1{10} r\). Choose an \(n \in \N\) with
  \(n > 1/\delta\). For each \(i = 0, \dotsc, n\), choose a rational
  \(q_i \in
    \intoo{f(\tfrac in) - \tfrac 1{10} r, f(\tfrac in) + \tfrac 1{10}r}
    \cap \Q\), using density of \(\Q\).
  Let \(g = \theta_n^{-1}(q_0, \dotsc, q_n)\).
  Now suppose \(x \in \intcc{0, 1}\) is arbitrary. Pick an
  \(i \in \set{0, \dotsc n - 1}\) and a \(t \in \intcc{0, 1}\) such that
  \(x = \tfrac{i + t}n\). Then
  \begin{align*}
   \abs{g(x) - f(x)}
    &= \abs{g(\tfrac{i + t}n) - f(\tfrac{i + t}n)} \\
    &= \abs{(1 - t)g(\tfrac in) + tg(\tfrac{i + 1}n) - f(\tfrac{i + t}n)} \\
    &= \abs{(1 - t)q_i + tq_{i + 1} - f(\tfrac{i + t}n)} \\
    &\le \abs{q_i - f(\tfrac{i + t}n)}
      + t\abs{q_{i + 1} - q_i} \\
    &\le \abs{q_i - f(\tfrac in)} + \abs{f(\tfrac in) - f(\tfrac{i + t}n)}
      + \abs{q_{i + 1} - q_i} \\
    &\le \tfrac 15 r
      + \abs{q_{i + 1} - q_i} \\
    &\le \tfrac 15 r
      + \abs{q_{i + 1} - f(\tfrac{i + 1}n)}
      + \abs{f(\tfrac{i + 1}n) - f(\tfrac in)}
      + \abs{f(\tfrac in) - q_i} \\
    &\le \tfrac 15 r + \tfrac 3{10} r \le \tfrac 12 r
  \end{align*}
  and it follows that \(d_\infty(f, g) \le \tfrac 12 r < r\).

  So indeed \(g \in D_r(f) \intersect A\), and \(A\) must be dense.

  \(B\) is clearly closed, since it is a closed ball with respect to the
  uniform metric.

  To show that \(B\) is not compact, it suffices to show that \(B\) is not
  sequentially compact. Define
  \begin{equation*}
   f_n(x) =
   \begin{cases*}
    0 & if \(0 \le x \le \tfrac 1{n + 1}\) \\
    1 & if \(x = \tfrac 12(\tfrac 1n + \tfrac 1{n + 1})\) \\
    0 & if \(\tfrac 1n \le x \le 1\) \\
    \text{piecewise linear}
     & on \(\intco{\tfrac 1{n + 1}, \tfrac 12(\tfrac 1{n + 1} + \tfrac 1n)}\)
       and \(\intoc{\tfrac 12(\tfrac 1{n + 1} + \tfrac 1n), \tfrac 1n}\)
   \end{cases*}
  \end{equation*}
  Then \(d_\infty(f_n, f_m) = \delta_{nm}\), so in fact the sequence
  \((f_n)\) has no uniformy Cauchy subsequence, and certainly no uniformly
  convergent subsequence.
  % left as exercise
\end{enumerate}

\section*{Appendix}

% \includepdf[pages=10-12]{izaakvd_AT_3_orig.pdf}
% \includepdf[pages=15-17]{izaakvd_AT_3_orig.pdf}

\end{document}
